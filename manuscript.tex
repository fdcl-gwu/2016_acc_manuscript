%%%%%%%%%%%%%%%%%%%%%%%%%%%%%%%%%%%%%%%%%%%%%%%%%%%%%%%%%%%%%%%%%%%%%%%%%%%%%%%%
%2345678901234567890123456789012345678901234567890123456789012345678901234567890
%        1         2         3         4         5         6         7         8

\documentclass[letterpaper, 10 pt, conference]{ieeeconf}  % Comment this line out if you need a4paper
%\documentclass[a4paper, 10pt, conference]{ieeeconf}      % Use this line for a4 paper
\IEEEoverridecommandlockouts                              % This command is only needed if 
                                                          % you want to use the \thanks command
\overrideIEEEmargins                                      % Needed to meet printer requirements.

% See the \addtolength command later in the file to balance the column lengths
% on the last page of the document

\usepackage{ieee_packages}

\title{\LARGE \bf
Constrained Attitude Control
}


\author{Shankar Kulumani and Taeyoung Lee$^*$
 \thanks{Shankar Kulumani and Taeyoung Lee are with Department of Aerospace Engineering, George Washington University, Washington, DC. Email: {\tt\footnotesize \{skulumani, tylee\}@gwu.edu }}
 }



\begin{document}

\maketitle
\thispagestyle{empty}
\pagestyle{empty}


%%%%%%%%%%%%%%%%%%%%%%%%%%%%%%%%%%%%%%%%%%%%%%%%%%%%%%%%%%%%%%%%%%%%%%%%%%%%%%%%
\begin{abstract}

200-300 summary of the paper

general topic area

new developments/contributions of the paper

relation/benefits over previous work

summary of the results and contributions

Awesome abstract

\end{abstract}


%%%%%%%%%%%%%%%%%%%%%%%%%%%%%%%%%%%%%%%%%%%%%%%%%%%%%%%%%%%%%%%%%%%%%%%%%%%%%%%%
\section{INTRODUCTION}

Spacecraft are complicated electro mechanical devices and have become critical to national security.
Spacecraft operations require extensive planning periods and human interaction to perform.
The ability to robustly and autonomously plan and execute complicated spacecraft maneuvers can drastically improve the performance and duty cycle of assets.
One key technology to enable spacecraft autonomy is the execution of large angular slews in the presence of constraints.
For example, many spacecraft missions are equipped with sensitive optical instruments that require targeting while simultaneously avoiding direct exposure to sunlight.
One of the distinct features of the attitude dynamics of a rigid body is that it evolves on a nonlinear manifold.
The special orthogonal group, \( \SO \), yields unique properties that are not present in dynamic systems on a linear space. 
In addition, the removal of constrained zones from the rotational configuration space results in a non-convex and more complicated system.
The attitude dynamics and control of spacecraft without constraint has been extensively studied in the literature.
This work builds upon previous results to include attitude constraints on \( \SO \).

In this work an attitude control system is presented to track a desired trajectory while avoiding constrained regions. 
The control system is formulated directly on the nonlinear manifold \( \SO \) and avoids the issues associated with minimal representations.
The approach is based on a logarithmic barrier potential function developed directly on \( \SO \). 
This avoids the singularities associated with Euler angles~\cite{mcinnes1994}.
Euler angles are not suitable for large angle slew maneuvers as the type of Euler angle sequence should be switched to avoid the singular regions.
Quaternions do not suffer from singularities but as the three-sphere double covers the special orthogonal group, there exists two antipodal points on the three-sphere which represents an equivalent attitude.
As a result, control schemes based on quaternions should carefully resolve this ambiguity to avoid unwinding, where a rigid body unnecessarily rotates through a large angle in spite of a small initial error.~\cite{lee2011b}.

introduction to attitude dynamics problem

Attitude dynamics evolve on a nonlinear manifold, unique properties not experienced in linear systems

difficulties of attitude control (topological restriction, nonlinear manifold)

difficulties with attitude representations.

Linear vector space used to represent motion on nonlinear manifold. 

Euler angles/MRP have singularities not suitable for large angular rotations

Quaterions have no singularities but double coverage as the three sphere double covers the special orthogonal group.
There are two equivalent quaternions that represent the same attitude.

Discuss attitude control in the presence of constraints. 
Key technology/capability to allow for autonomous operation of spacecraft and UAVs
Sensitive optical instruments that require reorientation while avoiding exposure to the sun or bright sources

Removal of constrained regions from the configuration space results in a nonconvex space/region.

Benefits of this approach over previous work ( cite papers).
Use of artifical potential functions
Mcinnes looked at reorientation using euler angles. 
Singularities would become evident for arbitrarily large angular slews

Determine manuever a priori using geometric/kinematic relationships

Computational based approaches.
Randomized algorithms or direct optimal control

\section{PROBLEM FORMULATION}

Describe attitude dynamics EOMS.
Explain attitude representation using \SO.
Hat and vee map properties used in developments

Configuration error function with attractive and repulsive terms
Define the various attitude error functions and where they are derived from
summarize some of the properties of the attractive error function from previous papers
Show the development of the configuration error terms and equilibrium structure
\section{ATTITUDE STABILIZATION}

Derive error dynamics
Introduce control input function and show that it is stable without any disturbance

Add the disturbance term and introduce adaptive update law
Mention how for a fixed disturbance this is integral control 
Go through the various developments of proving the disturbance case is stable for correct choice of gains

\section{NUMERICAL EXAMPLE}
Parameters used in the simulation

Show stabilization with and without repulsive term
Show stabilization with and without adaptive update law

Plot of responses

Get the configuration error function mapped onto the attitude sphere

Show rotation animation of spacecraft avoiding constrained regions
\section{EXPERIMENT}

Discuss experimental setup of the hexrotor on spherical joint

Describe aim of experiment. 
Demonstrate that hexrotor fixed sensor can stabilize to desired attitude while avoiding an obstacle.
Mount a sphere at an appropriate distance such that the subtended angle is equal to the cone constraint

Show plots of response with and without avoidance term
Response should be similar to numerical simulation

\section{CONCLUSIONS}

Discuss how this method allows for on board implementation of attitude avoidance during reorientations

Improves upon previous work by extending to the full nonlinear manifold representation and avoids quaternion ambiguities

Proven stable and validated via hardware experiments

\addtolength{\textheight}{-12cm}   % This command serves to balance the column lengths
                                  % on the last page of the document manually. It shortens
                                  % the textheight of the last page by a suitable amount.
                                  % This command does not take effect until the next page
                                  % so it should come on the page before the last. Make
                                  % sure that you do not shorten the textheight too much.

%%%%%%%%%%%%%%%%%%%%%%%%%%%%%%%%%%%%%%%%%%%%%%%%%%%%%%%%%%%%%%%%%%%%%%%%%%%%%%%%



%%%%%%%%%%%%%%%%%%%%%%%%%%%%%%%%%%%%%%%%%%%%%%%%%%%%%%%%%%%%%%%%%%%%%%%%%%%%%%%%



%%%%%%%%%%%%%%%%%%%%%%%%%%%%%%%%%%%%%%%%%%%%%%%%%%%%%%%%%%%%%%%%%%%%%%%%%%%%%%%%
\section*{APPENDIX}

Detail on finding bounds of \( \norm{\dot{e_R}} \)

Stability proofs

Further discussion on equilibrium positions and structure of configuration error function

\section*{ACKNOWLEDGMENT}



%%%%%%%%%%%%%%%%%%%%%%%%%%%%%%%%%%%%%%%%%%%%%%%%%%%%%%%%%%%%%%%%%%%%%%%%%%%%%%%%


\bibliography{library}
\bibliographystyle{IEEEtran}

\end{document}
